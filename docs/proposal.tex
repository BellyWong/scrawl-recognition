% Proposal for the features and scoring and stuff of my senior project.
\documentclass{article}
\title{Senior Project Proposal}
\author{Steve Jarvis}
\date{\today}
% Disables chapter and section numbering
\setcounter{secnumdepth}{-1} 

\begin{document}
\maketitle

% Describe what the project is.
\section{What It Is...}
	\paragraph{}The project will be an application to recognize handwritten characters.  The
	focus and functional block of the project is a neural network and I plan to 
	design it for universal use, versus tailored specifically for handwriting 
	recognition.
	\paragraph{}The network will be continually training on euclid against the MNIST database
	of handwritten digits\footnote{Info for the MNIST Database can be found at: 
	yann.lecun.com/exdb/mnist/}. I'll have an iOS front end on which to write, 
	that will query a web page on euclid and return the results using the best 
	known weights configuration at the time. It will be constantly improving in 
	the background, the user will always see the best results possible.
	\paragraph{}This project will focus specifically on digit recognition (0-9) but including 
	letters--or any other symbols--would only require an amended data set on which
	to train and an appropriately scaled neural network.
	
% What technologies it uses
\section{Technologies...}
	\begin{itemize} \itemsep -2 pt
		\item Python
		\item Python CGI
		\item Objective C
		\item iOS
		\item JSON
		\item HTTP
		\item Calculus\footnote{Maybe not a technology, but first time I’ve used it writing software.}
		\item \LaTeX\footnote{Not necessary, but using it to write these docs. It's kinda cool.}
	\end{itemize}
	
\section{What I Hope to Learn...}
	\paragraph{}This is a venture into machine learning, and I hope to learn just that. All the 
	applications I sent to graduate schools expressed interest in AI and machine learning, but 
	I don’t yet know how to write anything of substance in the fields. I’m hoping this is a 
	challenging and meaningful entrance to the subject area.

	\paragraph{}I will learn some iOS development as well.

\section{Points on Points...}
	\subsection{Neural Network - 15 points}
		\begin{itemize} \itemsep -2 pt
			\item Scalable size for universal use - 5 pts
			\item Learn and implement a back-progagated training algorithm - 5 pts
			\item Option for momentum during training\footnote{Momentum coupled with the
					learning rate can reduce time for training.} - 2 pts
			\item Use a smooth, symmetrical sigmoidal function to determine activation 
					(and by extension, its derivative for back-propagation) - 2 pts
			\item Ability to save/load weights. - 1 pts
		\end{itemize}
	\subsection{Network Training App - 12 points}
		\begin{itemize} \itemsep -2 pt
			\item Automatically store weights when new best performance is achieved - 3 pts
			\item Automatically load best known weights on start of training - 2 pts
			\item Include experimental mode, which will graph performance of various 
					configurations, to help determine optimal settings of production network - 5 pts
			\item Option for verbose/silent - 1 pts
			\item Include a log. Important since running unattended - 1 pts
		\end{itemize}
	\subsection{iOS App – 8 pts}
		\begin{itemize} \itemsep -2 pt
			\item 3 pts - Automatically recognize a pause in writing and submit the image for recognition.
			\item 3 pts – Display percent certainty.
			\item 1 pts - Graceful notification on loss of network.
			\item 1 pts – Optionally display grid for visual estimate of what data network receives.
		\end{itemize}
	\subsection{Web Page – 5 pts}
		\begin{itemize} \itemsep -2 pt
			\item 3 pts – Find and load the latest weights after initializing a neural network.
			\item 2 pts – Output JSON feedback including both guesses and associated certainty.
		\end{itemize}
	\subsection{Total - 40 pts}
	
\section{Grading...}
0-20 = Failure \\
20-24 = D \\
25-29 = C \\
30-34 = B \\
35-40 = A \\

\end{document}